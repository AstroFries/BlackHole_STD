\documentclass[a4paper, 12pt]{article}
\usepackage{ctex} %使文档可以输入中文
\usepackage{amsfonts}
\usepackage{float}
\usepackage{graphicx}
\usepackage{amsmath,amssymb,amsfonts}
\begin{document}
    \title{黑洞渲染器}
    \maketitle
    \section{简介}
    电影《星际穿越》中名为“卡冈图雅”的黑洞给了我很大的震撼,从第一次看到它我就想自己复现一遍。在24秋期末
    看到有一个别的团队做了一个效果不错的,这让我重新想起之前的计划,于是25春选了计算机图形学、开始
    自己看广义相对论,准备完成这个项目。感谢大作业给了我一个这样的机会(不然可能拖到暑假自己
    慢悠悠地做了)。

    本项目主要使用cpp编写,采用CMake构建,调用库包含Eigen、GLAD、GLFW、ImGui、GSL以及Matplotlibcpp。最后两个
    主要用于测试(GSL为了显示氢原子的轨道以测试体渲染,Matplotlibcpp以测试测地线方程求解器)。以及部分glsl语言实现
    渲染等效果。项目也包含了
    部分python语言编写的符号计算程序(用于计算任意时空下的联络以便计算测地线)。

    本项目使用的环境是VSCode,CMake配置使用Visual Studio Community 2022 Release - amd64.
    如果您在运行GEODESIC\underline \ TEST或GEODESIC\underline \ AUTO\underline \ TEST项目,请确保
    您的环境中存在python以及numpy.
    \section{实现流程}
    
    
    
\end{document}