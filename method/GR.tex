\documentclass[a4paper, 12pt]{article}
\usepackage{ctex} %使文档可以输入中文
\usepackage{amsfonts}
\usepackage{float}
\usepackage{graphicx}
\usepackage{amsmath,amssymb,amsfonts}
\begin{document}
    \title{非平直时空光学}
    \maketitle 
    \section{测地线}
    在广义相对论中,自由物体的运动是时空中的一条测地线,遵循测地线方程:
    \begin{align*}
        \frac{\mathrm d^2 x^{\mu}}{\mathrm d \lambda ^2} = {\Gamma^{\mu}}_{\nu\eta}\frac{\mathrm d x^{\nu}}{\mathrm d \lambda}\frac{\mathrm d x^{\eta}}{\mathrm d \lambda} ,\mu = 0,1,2,3 
    \end{align*}
    光比较特殊,光在时空的轨迹上切矢的模长恒为零。
    \begin{align*}
        g_{\mu\nu}\frac{\mathrm d x^{\mu}}{\mathrm d \lambda}\frac{\mathrm dx^{\nu}}{\mathrm d\lambda} = 0  
    \end{align*}
    \section{施瓦西时空}
    施瓦西时空的线元表达式如下:
    \begin{align*}
        \mathrm ds^2 = -(1-\frac{2M}{r} )\mathrm dt^2 +(1-\frac{2M}{r} )^{-1}\mathrm dr^2 + r^2\mathrm d\theta^2 + r^2\sin^2 \theta \mathrm d\phi ^2
    \end{align*}
    施瓦西时空具有球对称性,因此周遭光的运动轨迹一直在同一个平面内,这个平面过黑洞中心。\\
    用$\left\{r,\theta,\phi,t\right\}$坐标描述时空,不妨将光放在$\theta = \frac{\pi}{2} $平面上,则光的轨迹可以被化简到如下方程:
    \begin{align*}
        \frac{\mathrm d^{2}}{\mathrm d\phi^{2}} (\frac{1}{r}) + \frac{1}{r} = 3M(\frac{1}{r} )^2
    \end{align*}
    \section{非平直时空下的光通截面}
    我们知道在平直时空下光的辐射强度和发射距离平方成反比,而人眼看到的大小也和距离平方成反比。\\
    现在想研究光通截面的面积究竟在光线轨迹上是如何变化的。\\
    研究两个相互靠近的光子的运动,设他们在时空中的参数曲线分别为$\mathbf x(\lambda),\mathbf x(\lambda)+\widetilde {\mathbf x}(\lambda) $   ,其中$\lambda$ 是仿射参数。
    他们各自遵守测地线方程:
    \begin{align*}
        \frac{\mathrm d^2 x^{\mu}}{\mathrm d \lambda ^2} = {\Gamma^{\mu}}_{\nu\eta}\frac{\mathrm d x^{\nu}}{\mathrm d \lambda}\frac{\mathrm d x^{\eta}}{\mathrm d \lambda} ,\mu = 0,1,2,3 
    \end{align*}
    \begin{align*}
        \frac{\mathrm d^2 (x^{\mu}+\widetilde{\mathbf x}^{\mu})}{\mathrm d \lambda ^2} = {\Gamma'^{\mu}}_{\nu\eta}\frac{\mathrm d (x^{\nu}+\widetilde{\mathbf x}^{\nu})}{\mathrm d \lambda}\frac{\mathrm d (x^{\eta}+\widetilde{\mathbf x}^{\eta})}{\mathrm d \lambda} ,\mu = 0,1,2,3 
    \end{align*}
    因为相互靠近,可以将$\widetilde{\mathbf x}$视为相对$\mathbf x$的小量
    带入测地线,相减得到:
    \begin{align*}
        \frac{\mathrm d^2 \widetilde{\mathbf x}^{\mu}}{\mathrm d\lambda ^2} = 2{\Gamma ^{\mu}}_{\nu\eta}\frac{\mathrm d x^{\nu}}{\mathrm d\lambda}\frac{\mathrm d \widetilde{\mathbf x}^\eta}{\mathrm d\lambda}  
        +{\Gamma ^{\mu}}_{\nu\eta,\kappa}\frac{\mathrm d x^{\nu}}{\mathrm d \lambda}\frac{\mathrm d x^{\eta}}{\mathrm d \lambda}\widetilde{\mathbf x}^{\kappa}
    \end{align*}
    令:
    \[{A^{\mu}}_{\eta}(\lambda) = -2{\Gamma ^{\mu}}_{\nu\eta}\frac{\mathrm d x^{\nu}}{\mathrm d\lambda}\]
    \[{B^{\mu}}_{\eta}(\lambda) = -{\Gamma ^{\mu}}_{\nu\kappa,\eta}\frac{\mathrm d x^{\nu}}{\mathrm d \lambda}\frac{\mathrm d x^{\kappa}}{\mathrm d \lambda}\]
    都是通过$\mathbf x(\lambda)$能计算出来的矩阵,得到方程:
    \begin{align*}
        \frac{\mathrm d^2 \widetilde{\mathbf x}^{\mu}}{\mathrm d\lambda ^2} +{A^{\mu}}_{\eta}(\lambda)\frac{\mathrm d \widetilde{\mathbf x}^\eta}{\mathrm d\lambda} +{B^{\mu}}_{\eta}(\lambda)\widetilde{\mathbf x}^\eta = 0
    \end{align*}
    也就是说如果一开始发射一堆光子,我们可以通过上式计算光子运动到任意位置的密度。
    

    黑体辐射、光谱和RGB的转换。
    黑体辐射遵循如下公式:

    而人眼看到的RGB三分量可以看作辐射谱在三个基函数下的坐标。

\end{document}